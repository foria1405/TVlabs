\documentclass[12pt]{article}
\usepackage[utf8]{inputenc}
\usepackage[russian]{babel}
\usepackage [left=30 mm, top=30 mm, right=30 mm, bottom=20mm, nohead, footskip=10 mm] {geometry}
\usepackage{pscyr}
\usepackage[T2A]{fontenc}
\usepackage{amsmath}
\usepackage{multirow}
\usepackage{xcolor}
\usepackage{amssymb}
\usepackage{graphicx}
\graphicspath{{src/}}
\usepackage{listings}   
\usepackage{hyperref}
\usepackage{fancyhdr}
%\usepackage{algorithm}
\usepackage{algpseudocode}
\usepackage{indentfirst}
\usepackage{listings}
\usepackage{float}%"Плавающие" картинки
\hypersetup{
    colorlinks=true,
    linkcolor=blue,
    filecolor=magenta,      
    urlcolor=cyan,
    pdftitle={Sharelatex Example},
    bookmarks=true,
    pdfpagemode=FullScreen,
}
\usepackage{wrapfig}%Обтекание фигур (таблиц, картинок и прочего)

\parindent=24pt
\newtheorem{theorem}{Theorem}

\begin{document}

\begin{center}
\hfill \break
\large{МИНОБРНАУКИ РОССИИ} \\
\hfill \break
\small {ФЕДЕРАЛЬНОЕ ГОСУДАРСТВЕННОЕ БЮДЖЕТНОЕ ОБРАЗОВАТЕЛЬНОЕ УЧРЕЖДЕНИЕ }\\
\small { ВЫСШЕГО ПРОФЕССИОНАЛЬНОГО ОБРАЗОВАНИЯ  } \\
\hfill \break
\normalsize {\textbf{ <<САНКТ-ПЕТЕРБУРГСКИЙ ПОЛИТЕХНИЧЕСКИЙ УНИВЕРСИТЕТ } }\\
{\normalsize {\textbf { ПЕТРА ВЕЛИКОГО>>}}} \\
\hfill \break
\large{Институт Прикладной математики и механики }\\
\hfill \break
\large{ Кафедра: <<Телематика ( при ЦНИИ РТК )>> }\\
\hfill \break
Направление 02.03.01 Математика и компьютерные науки\\
\vskip 1cm
\Large {Отчёт по дисциплине:}
\vskip 0.2cm
\Large{<<Теория вероятностей и математическая статистика >>} \\
\hfill \break
\large{Лабораторная работа № 7} \\
\hfill \break
\large{<<Метод максимального правдоподобия. Проверка гипотез о распределении по критерию хи-квадрат.>>} \\
\hfill \break
\vskip 0.3cm
\vskip 0.5cm
\end{center}


\begin {tabular}{cccc}
\hspace{0.5cm}Обучающийся: &\underline {\hspace{3cm}} &  &Фомина Дарья Дмитриевна \\\\
\hspace{0.5cm}Руководитель: &\underline {\hspace{3cm}} & &Баженов Александр Николаевич\\\\
\end{tabular}
\vskip 1.5 cm
\hspace{9cm}\def \hrf#1{\hbox to#1{\hrulefill}}<<\hrf{2em}>>  \hrf{6em}  20\hrf{1em}~r.
\vskip 1.5cm
\begin {center} Санкт-Петербург 2021 \end{center}

\thispagestyle{empty}

\newpage


\tableofcontents


\newpage
\section{Постановка задачи}

\begin{itemize}
	\item Сгенерировать выборку объёмом 100 элементов для нормального распределения $N(0,1)$.
\item По сгенерированной выборке оценить параметры $\mu$ и $\sigma$ нормального закона методом максимального правдоподобия.
\item В качестве основной гипотезы $H_0$ считать, что сгенерированное распределение имеет вид $N(\hat{\mu}, \hat{\sigma})$, где $\hat{\mu}$ и $\hat{\sigma}$ оценки метода максимального правдоподобия.
\item Проверить основную гипотезу, используя критерий согласия $\chi^2$. В качестве уровня значимости взять $\alpha = 0.05$.
\item Исследовать точность (чувствительность) критерия $\chi^2$ -- сгенерировать выборки равномерного распределения и распределения Лапласа из 20 элементов. Проверить их на нормальность, то есть проверить, принимает ли критерий $\chi^2$ гипотезу, что элементы этих выборок распределены по закону $N(\hat{\mu}, \hat{\sigma})$.
\end{itemize}


\newpage
\section{Математическое описание}
\subsection{Метод максимального правдоподобия}
	\par Пусть $\{x_n\}$ -- случайная выборка из генеральной совокупности с плотностью вероятности $f(x, \theta)$, где $\theta \in \mathbb{R}^m, m \in \mathbb{N}$ -- совокупность параметров плотности вероятности.
\vskip 0.3cm
	\par \textbf{Функцией правдоподобия} назовём совместную плотность вероятность независимых случайных величин $x_1, ..., x_n$ с \textit{одним и тем же} параметром распределения $\theta$:
	\begin{equation}
	L(x_1, ..., x_n, \theta) = f(x_1, \theta) \cdot ... \cdot f(x_n, \theta)
	\end{equation}
\vskip 0.3cm
	\textbf{Оценкой максимального правдоподобия} $\hat{\theta_{мп}}$ назовём такое значение параметра, при котором функция правдоподобия достигает своего максимума.
\vskip 0.3cm
	Оценка максимального правдоподобия:
	\begin{equation}
		\hat{\theta}_{\text{МП}} = \arg\max L(x_1,\dots,x_n,\theta)
	\end{equation}
	Система уравнений правдоподобия (в случае дифференцируемости функции правдоподобия):
	\begin{equation}
		\frac{\partial L}{\partial \theta_k} = 0 \; \text{ или } \; \frac{\partial \ln L}{\partial \theta_k} = 0, \; k = 1,\dots,m.
		\label{difEquation}
	\end{equation}

	\subsection{Проверка гипотезы о законе распределения. Метод $\chi^2$}
	Выдвинута гипотеза $H_0$ о генеральном законе распределения с функцией распределения $F(x)$.\\
	\phantom{0}\\
	Рассматриваем случай, когда гипотетическая функция распределения $F(x)$ не содержит неизвестных параметров.\\
	\phantom{0}\\
	\textbf{Правило проверки гипотезы о законе распределения по методу $\chi^2$.}\cite{lit1}
	\begin{enumerate}
		\item Выбираем уровень значимости $\alpha$.

		\item Находим квантиль $\chi^2_{1-\alpha}(k-1)$ распределения $\chi^2$ с $k-1$ степенями свободы порядка $1-\alpha$.

		\item С помощью гипотетической функции распределения $F(x)$ вычисляем
		вероятности $p_i = P(X\in \Delta_i), i=1,\dots,k$.

		\item Находим частоты $n_i$ попадания элементов выборки в подмножества
		$\Delta_i, i = 1,\dots,k$.

		\item Вычисляем выборочное значение статистики критерия $\chi^2$ :
		\begin{equation}
 		\chi^2_B = \sum_{i=0}^{k} \frac{(n_i - np_i)^2}{np_i}.
 		\label{hi2}
 		\end{equation}
Данное обозначение критерия близости, называемого \textit{статистикой критерия $\chi^2$} неслучайно, поскольку имеет место следующая теорема:
	
\begin{theorem} \textbf{Пирсона}.
		Статистика $\chi^2$ асимптотически распределена по закону $\chi^2$ с $k-1$ степенями свободы.
\end{theorem}

То есть, какую бы мы гипотезу ни проверяли, функция распределения статистики стремится к истинной функции распределения случайной величины с плотностью вероятности:
\begin{equation}
	f_{k-1}(x)=
	\begin{cases}
	0, & x \leq 0 \\
	\frac{1}{2^{\frac{k-1}{2}}\Gamma\left({\frac{k-1}{2}}\right)}x^{\frac{k-3}{2}}e^{-\frac{x}{2}}, & x > 0
	\end{cases}
	\end{equation}

Также необходимо понять, какую гипотезу мы будем считать достоверной, а какую -- нет. Для этого необходимо ввести \textit{уровень значимости} $\alpha$.

		\item Сравниваем $\chi^2_B$ и квантиль  $\chi^2_{1-\alpha}(k-1)$.
		\begin{itemize}
			\item[$\text{а)}$] Если $\chi^2_B < \chi^2_{1-\alpha}(k-1)$, то гипотеза $H_0$ на данном этапе проверки принимается.
			\item[$\text{б)}$] Если $\chi^2_B \geq \chi^2_{1-\alpha}(k-1)$, то гипотеза $H_0$ отвергается, выбирается одно из альтернативных распределений, и процедура проверки повторяется.
		\end{itemize}
	\end{enumerate}
	Количество интервалов $k$ можно определить с помощью эвристики:
	\begin{equation}
		k \approx 1.72\cdot\sqrt[3]{n}
		\label{k}
	\end{equation}

\newpage
\section{Особенности реализации}
Программа для данной лабораторной работы была написана на языке Python 3.9. 
\par Для генерации выборок был использован модуль stats библиотеки scipy.  
	
\par В ходе данной лабораторной работы была реализована функция, с помощью которой находится количество промежутков разбиения в соответствии с эвристикой \eqref{k}.
	\vskip 0.3cm
\hrule width 16cm height 1pt
\begin{verbatim}	 
	def count_intervals(x, xmin, xmax):
    c = 0
    for v in x:
        c += 1 if xmin <= v < xmax else 0
    return c
\end{verbatim}	 
\hrule width 16cm height 1pt
    
	\vskip 0.5cm 
	Также в соответствии с формулой \eqref{hi2} были вычислены выборочные значения статистики критерия $\chi^2$ для заданных выборок.
	\vskip 0.3cm
\hrule width 16cm height 1pt
\begin{verbatim}	 
def table(sel, k, file):
    x = np.array(sel)
    x.sort()
    d = (x[-1] - x[0]) / k
    F = MLE(x)
    n = len(x)
    chi = 0
    for i in range(k):
        if i == 0:
            a0 = r'-\infty'
            a1 = round(x[0] + d, 2)
            p = round(F(a1), 4)
            ni = count_intervals(x, x[0] - 1, a1)
        elif i == k - 1:
            a0 = round(x[0] + d*i, 2)
            a1 = r'+\infty'
            p = round(1 - F(a0), 4)
            ni = count_intervals(x, a0, x[-1] + 1)
        else:
            a0 = round(x[0] + d*i, 2)
            a1 = round(a0 + d, 2)
            p = round(F(a1) - F(a0), 4)
            ni = count_intervals(x, a0, a1)
        s=f'\\hline{i+1}&$({tr(a0,2)};{tr(a1,2)})$&{ni}&{tr(p,4)}&'
        s += f' {tr(ni-n*p, 2)} & '
        dchi = round((ni-n*p)**2 / (n*p), 2)
        s += f'{tr(dchi, 4)} ' + r'\\'
        chi += dchi
        file.write(s + '\n')
    chi = round(chi, 2)
    return chi
\end{verbatim}
\hrule width 16cm height 1pt
    
\newpage

\section{Результаты работы программы}
\subsection{Метод максимального правдоподобия}
	С помощью метода максимального правдоподобия были получены следующие оценки:
	
	$$\hat{\mu} \approx 0.035$$$$ \hat{\sigma} \approx 0.992$$
	
\subsection{Критерий хи-квадрат}
Критерий согласия $\chi^2$ для выборки из 100 элементов по закону $N(x,0,1)$:
\begin{itemize}
	\item Количество промежутков $k = [1.72\sqrt[3]{n}+1] = 8$. (\ref{k})
	\item Уровень значимости $\alpha = 0.05$.
	\item Тогда квантиль из таблицы[3, с. 358] $\chi^2_{1-\alpha}(k-1) = \chi^2_{0.95}(7) = 14.067$.
\end{itemize}

	В таблице \ref{N(m, s)} представлены этапы вычисления выборочного значения  $\chi_{B}^2$.
	

		\begin{table}[H]
		\begin{center}
			\begin{tabular}{|l|l|l|l|l|l|}
			\hline
			$i$      & $\Delta_i$             & $n_i$ & $p_i$ & $n p_i$ & $\frac{(n_i - np_i)^2}{np_i}$ \\ \hline
			1        & $(-\infty, -1.49)$ & 8   & 0.070 & 7.02   & 0.01                        \\ \hline
			2        & $[-1.49, -1.05)$   & 9   & 0.084 & 8.35   & 0.04                          \\ \hline
			3        & $[-1.05, -0.59)$   & 7   & 0.132 & 13.16  & 2.91                         \\ \hline
			4        & $[-0.59, -0.15)$   & 20  & 0.169 & 16.94  & 0.57                        \\ \hline
			5        & $[-0.15, 0.3)$     & 18  & 0.178 & 17.82  & 0.00                     \\ \hline
			6        & $[0.3, 0.75)$      & 17  & 0.153 & 15.31  & 0.18                     \\ \hline
			7        & $[0.75, 1.20)$    & 13  & 0.108 & 10.75  & 0.45                         \\ \hline
			8       & $[1.20, \infty)$   & 8   & 0.106 & 10.62  & 0.64                  \\ \hline
			$\Sigma$ & -                      & 100   & 1.000  & 100.00   & {\bf 4.92}   \\  \hline                    
		\end{tabular}
		\end{center}
	\caption{Вычисление $\chi_{B}^2$ при проверке гипотезы $H_0$ о законе распределения $N(\hat{\mu}, \hat{\sigma})$ для выборки распределения $N(0, 1)$} \label{N(m, s)}
	\end{table}

	
	
	Выборочное значение $\chi_{B}^2 = 4.92$ меньше, чем табличное   $\chi_{0.95}^2(7)=14.067$, следовательно, на данном этапе гипотезу $H_0$ можно принять.
	

\newpage 
\subsection{Исследование на чувствительность критерия хи-квадрат}
\paragraph{Распределение Лапласа}
Рассмотрим гипотезу $H_0$ , что выборка из 20 элементов, распределенная по закону  $L(0,\frac{1}{\sqrt{2}})$, распределена по нормальному распределению, используя критерий согласия $\chi^2$:
\begin{itemize}
	\item Размер выборки $n=20$.
	\item Количество промежутков $k = [1.72\sqrt[3]{n}+1] = 5$.
	\item Уровень значимости $\alpha = 0.05$.
	\item Квантиль из таблицы[3, с. 358] $\chi^2_{1-\alpha}(k-1) = \chi^2_{0.95}(4) \approx 9.487$.
\end{itemize}

В Таблице 2 представлены этапы вычисления критерия хи-квадрат для проверки гипотезы о законе распределения.
	
		\begin{table}[H]
			\begin{center}
			\label{Lap}
		\begin{tabular}{|l|l|l|l|l|l|}
			\hline
			$i$      & $\Delta_i$             & $n_i$ & $p_i$ & $n p_i$ & $\frac{(n_i - np_i)^2}{np_i}$ \\ \hline
			1        & $(-\infty, -0.83)$ & 5  & 0.184 & 3.68 & 0.47                         \\ \hline
			2        & $[-0.83, -0.31)$   & 3  & 0.182 & 2.98 & 0.00                       \\ \hline
			3       & $[-0.31, 0.20)$    & 5  & 0.128 & 3.64 & 0.50                       \\ \hline
			4       & $[0.20, 0.72)$     & 2  & 0.179 & 3.58 & 0.69                         \\ \hline
			5       & $[0.72, \infty)$   & 5  & 0.306 & 6.11 & 0.20                         \\ \hline
			$\Sigma$ & -                      & 20    & 1.000  & 20.00    & {\bf1.88}                  \\ \hline
		\end{tabular}
	\end{center}
	
	\caption{Вычисление $\chi_{B}^2$ при проверке гипотезы $H_0$ о законе распределения $N(\hat{\mu}, \hat{\sigma})$ для выборки распределения $L(0,\frac{1}{\sqrt{2}})$}
	\end{table}
	

Выборочное значение $\chi_{B}^2 = 1.88$ меньше, чем табличное   $\chi_{0.95}^2(4)=9.487$, следовательно, на данном этапе гипотезу $H_0$ можно принять.
	
\paragraph{Равномерное распределение}
Рассмотрим гипотезу $H_0$ , что выборка из 20 элементов, распределенная по закону $U(x,-\sqrt(3),\sqrt{3})$, распределена по нормальному распределению, используя критерий согласия $\chi^2$:
\begin{itemize}
	\item Размер выборки $n=20$.
	\item Количество промежутков $k = [1.72\sqrt[3]{n}+1] = 5$.
	\item Уровень значимости $\alpha = 0.05$.
	\item Квантиль из таблицы [3, с. 358] $\chi^2_{1-\alpha}(k-1) = \chi^2_{0.95}(4) \approx 9.487$.
\end{itemize}

В Таблице 3 представлены этапы вычисления критерия хи-квадрат для проверки гипотезы о законе распределения.
	

		\begin{table}[H]
			\begin{center}
				\begin{tabular}{|l|l|l|l|l|l|}
				\hline
				$i$      & $\Delta_i$             & $n_i$ & $p_i$ & $n p_i$ & $\frac{(n_i - np_i)^2}{np_i}$ \\ \hline
				1        & $(-\infty, -0.88)$ & 5  & 0.171 & 3.42 & 0.73                        \\ \hline
				2        & $[-0.88, -0.18)$   & 2  & 0.207 & 4.13 & 1.10                         \\ \hline
				3        & $[-0.18, 0.52)$    & 4  & 0.250 & 5.00 & 0.20                       \\ \hline
				4        & $[0.52, 1.22)$     & 4  & 0.204 & 4.09 & 0.00                     \\ \hline
				5        & $[1.22, \infty)$   & 5  & 0.167 & 3.34 & 0.82                         \\ \hline
				$\Sigma$ & -                      & 20    & 1.000  & 20.00    &{\bf 2.86}                  \\ \hline
			\end{tabular}
			\end{center}
	\label{U}
	\caption{Вычисление $\chi_{B}^2$ при проверке гипотезы $H_0$ о законе распределения $N(\hat{\mu}, \hat{\sigma})$ для выборки распределения $U(-\sqrt(3), \sqrt{3})$}
	\end{table}
	
	
		Выборочное значение $\chi_{B}^2 = 2.86$ меньше, чем табличное   $\chi_{0.95}^2(4)=9.487$, следовательно, на данном этапе гипотезу $H_0$ можно принять.

\newpage
 \section*{Заключение}
{\addcontentsline {toc}{section}{Заключение}}

В ходе данной лабораторной работы для была сгенерирована выборка для нормального распределения. С помощью метода максимального правдоподобия для полученной выборки нормального распределения были найдены оценки параметров $\mu$ и $\sigma$ для функции распределения $N(\mu, \sigma)$. 
\par Было установлено, что о.м.п. матожидания нормального распределения равна средневыборочному значению, а о.м.п. среднеквадратического отклонения вычисляется как стандартное выборочное отклонение. 
\par Также с помощью критерия $\chi^2$ была принята гипотеза о соответствии нормальной выборке нормальному распределению.
	
	Кроме того, в ходе данной лабораторной работы были сгенерированы выборки размером 20 элементов равномерного распределения и распределения Лапласа. Для полученных выборок с помощью критерия $\chi^2$ была проверена гипотеза о соответствии нормальной выборки данным распределениям. 
	Проверка чувствительности критерия хи-квадрат на малых выборках показала, что этот критерий может давать неверные результаты, то есть является не чувствительным на малых выборках, так как он не отверг гипотезу о нормальности распределения $U(-\sqrt{3}, \sqrt{3})$ и $L(0, \frac{1}{\sqrt{2}})$.
	
Точность оценки критерием $\chi^2$ напрямую зависит от размера выборки, так как по теореме Пирсона, статистика критерия распределена асимптотически. То есть размер выборки предполагается достаточно большим.

\newpage





\addcontentsline{toc}{section}{Список используемой литературы}
\begin{thebibliography}{}
	\bibitem{lit1}  Вероятностные разделы математики. Учебник для бакалавров технических направлений.// Под ред. Максимова Ю.Д. — Спб.: «Иван Федоров», 2001. — 592 c., илл.
	\bibitem{lit2}  Вентцель Е.С. Теория вероятностей: Учеб. для вузов. — 6-е изд. стер. — М.: Высш. шк., 1999.— 576 c.
	\bibitem{lit3}  Максимов Ю.Д. Математика. Теория и практика по математической статистике. Конспект-справочник по теории вероятностей : учеб. пособие / Ю.Д. Максимов; под ред. В.И. Антонова. — СПб. : Изд-во Политехн. ун-та, 2009. — 395 с. (Математика в политехническом университете).
\end{thebibliography}
\newpage
\appendix


\section{Приложение}
\label{sec:A}
\par Исходный код программы размещен на сервисе GitHub.
\par Ссылка на репозиторий: \url{https://github.com/foria1405/TVlabs}
\end{document}