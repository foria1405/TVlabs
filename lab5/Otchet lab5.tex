\documentclass[12pt]{article}
\usepackage[utf8]{inputenc}
\usepackage[russian]{babel}
\usepackage [left=30 mm, top=30 mm, right=30 mm, bottom=20mm, nohead, footskip=10 mm] {geometry}
\usepackage{pscyr}
\usepackage[T2A]{fontenc}
\usepackage{amsmath}
\usepackage{multirow}
\usepackage{xcolor}
\usepackage{amssymb}
\usepackage{graphicx}
\graphicspath{{src/}}
\usepackage{listings}   
\usepackage{hyperref}
\usepackage{fancyhdr}
%\usepackage{algorithm}
\usepackage{algpseudocode}
\usepackage{indentfirst}
\usepackage{listings}
\usepackage{float}%"Плавающие" картинки
\hypersetup{
    colorlinks=true,
    linkcolor=blue,
    filecolor=magenta,      
    urlcolor=cyan,
    pdftitle={Sharelatex Example},
    bookmarks=true,
    pdfpagemode=FullScreen,
}
\usepackage{wrapfig}%Обтекание фигур (таблиц, картинок и прочего)

\parindent=24pt


\begin{document}

\begin{center}
\hfill \break
\large{МИНОБРНАУКИ РОССИИ} \\
\hfill \break
\small {ФЕДЕРАЛЬНОЕ ГОСУДАРСТВЕННОЕ БЮДЖЕТНОЕ ОБРАЗОВАТЕЛЬНОЕ УЧРЕЖДЕНИЕ }\\
\small { ВЫСШЕГО ПРОФЕССИОНАЛЬНОГО ОБРАЗОВАНИЯ  } \\
\hfill \break
\normalsize {\textbf{ <<САНКТ-ПЕТЕРБУРГСКИЙ ПОЛИТЕХНИЧЕСКИЙ УНИВЕРСИТЕТ } }\\
{\normalsize {\textbf { ПЕТРА ВЕЛИКОГО>>}}} \\
\hfill \break
\large{Институт Прикладной математики и механики }\\
\hfill \break
\large{ Кафедра: <<Телематика ( при ЦНИИ РТК )>> }\\
\hfill \break
Направление 02.03.01 Математика и компьютерные науки\\
\vskip 1cm
\Large {Отчёт по дисциплине:}
\vskip 0.2cm
\Large{<<Теория вероятностей и математическая статистика >>} \\
\hfill \break
\large{Лабораторная работа № 5} \\
\hfill \break
\large{<<Двумерное нормальное распределение: выборочные коэффициенты корреляции и эллипсы рассеивания>>} \\
\hfill \break
\vskip 0.3cm
\vskip 0.5cm
\end{center}


\begin {tabular}{cccc}
\hspace{0.5cm}Обучающийся: &\underline {\hspace{3cm}} &  &Фомина Дарья Дмитриевна \\\\
\hspace{0.5cm}Руководитель: &\underline {\hspace{3cm}} & &Баженов Александр Николаевич\\\\
\end{tabular}
\vskip 1.5 cm
\hspace{9cm}\def \hrf#1{\hbox to#1{\hrulefill}}<<\hrf{2em}>>  \hrf{6em}  20\hrf{1em}~r.
\vskip 1.5cm
\begin {center} Санкт-Петербург 2021 \end{center}

\thispagestyle{empty}

\newpage


\tableofcontents


\newpage
\section{Постановка задачи}
 Дано двумерное нормальное распределение $N(x, y, 0, 0, 1, 1, \rho)$.

Требуется сгенерировать двумерные выборки размером 20, 60, 100 элементов и коэффициентами корреляции 0, 0.5, 0.9.

Каждую выборку необходимо сгенерировать 1000 раз и вычислить для неё:
\begin{itemize}
	\item Среднее значение
	\item Среднее значение квадрата
	\item Дисперсию коэффициента корреляции Пирсона
	\item Дисперсию коэффициента корреляции Спирмена
	\item Дисперсию квадрантного коэффициента корреляции
\end{itemize}

Также требуется произвести эти вычисления для смеси нормальных распределений:
\begin{equation}\label{mix}
f(x, y) = 0.9N(x, y, 0, 0, 1, 1, 0.9) + 0.1N(x, y, 0, 0, 10, 10, -0.9)
\end{equation}
Изобразить сгенерированные точки на плоскости и эллипсе равновероятности.
\newpage
\section{Математическое описание}
\subsection{Двумерное нормальное распределение}

Двумерная случайная величина (X,Y) называется {\it нормально распределённой}, если её плотность вероятности определена следующим образом:
\begin{multline}\label{N2}
	N(x, y, \overline{x}, \overline{y}, \sigma_x, \sigma_y, \rho) =
	\frac{1}{2 \sigma_x \sigma_y \sqrt{1 - \rho^2}} \times \\	
	\times exp \left( - \frac{1}{2(1 - \rho^2)} \left[ \frac{(x - \overline{x})^2}{\sigma_x^2} - 2 \rho (x - \overline{x}) (y - \overline{y}) + \frac{(y - \overline{y})^2}{\sigma_y^2}) \right] \right) 
\end{multline}

При этом компоненты $X$ и $Y$ двумерной нормальной случаной величины также распределены нормально с математическими ожиданиями $\overline{x}, \overline{y}$  и средними квадратическими отклонениями соответственно $\sigma_x$ и $\sigma_y$. \cite{lit1}

\subsection{Корреляционный момент(ковариация) и коэффициент корреляции}

\textit{Ковариацией} или \textit{корреляционным моментом} случайной величины называется матожидание произведения отклонений компонент случайной величины от её среднего:
\begin{equation}\label{K}
	K_{XY} = cov(X, Y) = M[(X - \overline{x})(Y - \overline{y})]
\end{equation}

\textit{Коэффициентом корреляции} является нормированный на единицу корреляционный момент. Показывает меру линейной зависимости между величинами. \cite{lit1}

\begin{equation}\label{rho}
	\rho = \frac{K_{XY}}{\sigma_x \sigma_y}
\end{equation}

\subsection{Выборочные коэффициенты корреляции}
\subsubsection{Выборочный коэффициент корреляции Пирсона}
Коэффициент корреляции, определяемый по выборочным данным, называется {\it выборочным коэффициентом корреляции}.
Для выборки двумерной случайной величины $\{x_i, y_i\}_{i=\overline{1,n}}$ наиболее естественным приближением корреляционного коэффициента является соотношение:

\begin{equation}\label{r}
	r = \frac{\frac{1}{n} \displaystyle \sum_{i=1}^{n}{\left(x_i - \overline{x}\right)\left(y_i - \overline{y}\right)}}{\sqrt{\displaystyle \sum_{i=1}^{n}{\frac{1}{n}\left(x_i - \overline{x}\right)^2\frac{1}{n}\left(y_i - \overline{y}\right)^2}}} = \frac{K_{XY}}{s_X s_Y},
\end{equation}
где $K_{XY}, s_X^2, s_Y^2$ -- выборочная ковариация и дисперсии соответствующих случайных величин. \cite{lit1}

\subsubsection{Выборочный квадрантный коэффициент корреляции}
Выборочный квадрантный коэффициент корреляции:

\begin{equation}\label{rq}
	r_Q=\frac{(n_1 + n_3) - (n_2 + n_4)}{n},
\end{equation}

где $n_1, n_2, n_3, n_4$ -- количества точек с координатами $(x_i, y_i)$, попавшими соответственно в I, II, III и IV квадранты декартовой системы с осями $x' = x - med \; x, y' = y - med \; y$ и с центром в точке с координатами $(med \; x, med \; y)$. \cite{lit1}

\subsubsection{Выборочный коэффициент ранговой корреляции Спирмена}

Обозначим ранги, соответствующие значениям переменной $X$, через $u$, а ранги, соответствующие значениям переменной $Y$, — через $v$.

        Выборочный коэффициент ранговой корреляции Спирмена :
       \begin{equation}
r_S = \frac{\frac{1}{n} \sum (u_i - \overline{u})(v_i - \overline{v})}{\sqrt{\frac{1}{n} \sum (u_i - \overline{u})^2 \frac{1}{n} \sum (v_i - \overline{v})^2}},
\label{rs}
\end{equation}

        где $\overline{u} = \overline{v} = \frac{1 + 2 + ... + n}{n} = \frac{n+1}{2}$ -- среднее значение рангов. \cite{lit1}
\subsection{Эллипсы рассеивания}

Уравнение проекции эллипса рассеивания на плоскость $xOy$:

    \begin{equation}
        \frac{(x-\overline{x})^2}{\sigma_x^2} - 2\rho\frac{(x-\overline{x})(y-\overline{y})}{\sigma_x\sigma_y} + \frac{(y-\overline{y})^2}{\sigma_y^2} = const.
        \label{ellipse}
    \end{equation}

Центр эллипса (\ref{ellipse}) находится в точке с координатами $(\overline{x}, \overline{y})$.
Произведя соответствующие выкладки, получаем, что большая полуось эллипса составляет с абсциссой угол, выраженный по следующей формуле:
\begin{equation}\label{tga}
	tg 2\alpha = \frac{2 \rho \sigma_x \sigma_y}{\sigma_x^2 - \sigma_y^2}
\end{equation}

Посмотрев на данное соотношение, можно понять, что угол поворота эллипса даёт нам качественное представление о степени коррелированности данных.

Контур эллипса является линией равной вероятности, поэтому такой эллипс называется эллипсом равной плотности либо {\it эллипсом рассеивания}. \cite{lit2}
\newpage
\section{Особенности реализации}
Программа для данной лабораторной работы была написана на языке Python 3.9. 

В данной лабораторной работе для генерации выборок двумерного распределения была использована библиотека scipy. Так же для построения эллипсов рассеивания была использована библиотека matplotlib. 

В ходе данной лабораторной работы были реализованы функции вычисления выборочных коэффицентов корреляции в соответствии с формулами \eqref{r}, \eqref{rq} и \eqref{rs}.  Так же была реализована функция, осуществляющая построение эллипсов рассеяния в соответствии с формулой \eqref{ellipse}.
\vskip 0.4cm
\par Функция для вычисления выборочного коэффициента корреляции Пирсона реализована по формуле \eqref{r}:
\hrule width 16cm height 1pt
\begin{verbatim}
def pearson(x, y):
    if len(x) != len(y):
        raise ValueError("Две выборки должны иметь одинаковый размер!")
    n = len(x)
    xm = sum(x) / n
    ym = sum(x) / n
    cov = sum([(x[i] - xm)*(y[i] - ym) for i in range(n)])
    sXsY = sqrt(sum([(x[i]-xm)**2 for i in range(n)]) * \
              sum([(y[i]-ym)**2 for i in range(n)]))
    return cov / sXsY
\end{verbatim}
\hrule width 16cm height 1pt
\vskip 0.4cm
\par Функция для вычисления выборочного квадрантного коэффициента корреляции реализована по формуле \eqref{rq}:
\hrule width 16cm height 1pt
\begin{verbatim}
def quadrant(x, y):
    if len(x) != len(y):
        raise ValueError("Две выборки должны иметь одинаковый размер!")
    xm = np.median(x)
    ym = np.median(y)
    sign = lambda x: 1 if x > 0 else -1 if x < 0 else 0
    return sum(map(lambda a: sign((a[0]-xm)*(a[1]-ym)), zip(x, y))) / len(x)
\end{verbatim}
\hrule width 16cm height 1pt
\vskip 0.4cm
\par Функция для вычисления выборочного коэффициента ранговой корреляции Спирмена реализована по формуле \eqref{rs}:
\hrule width 16cm height 1pt
\begin{verbatim}
def spearman(x, y):
    sx = sorted(x)
    sy = sorted(y)
    rx = [sx.index(a) for a in x]
    ry = [sy.index(a) for a in y]
    return pearson(rx, ry)
\end{verbatim}
\hrule width 16cm height 1pt
\vskip 0.4cm
\par Функция, осуществляющая построение эллипсов рассеяния в соответствии с формулой \eqref{ellipse}, в которой $const=9$:
\vskip 0.3cm
\hrule width 16cm height 1pt
\begin{verbatim}
def plotEllipse(x, y, ax, sigmas):
    r = pearson(x, y)
    ellipse = Ellipse((0, 0),
                      width=sqrt(1 + r),
                      height=sqrt(1 - r),
                      facecolor='none',
                      edgecolor='red')
    xm = sum(x) / len(x)
    ym = sum(y) / len(y)
    sx = sqrt(sum(map(lambda a: (a - xm)**2, x)) / len(x))
    sy = sqrt(sum(map(lambda a: (a - ym)**2, y)) / len(y))
    tr = transforms.Affine2D() \
         .rotate_deg(45) \
         .scale(sx * sigmas, sy * sigmas) \
         .translate(xm, ym)
    ellipse.set_transform(tr + ax.transData)
    ax.add_patch(ellipse)
    ax.scatter(x, y, s=0.9)    
        
    s = rvs2d(size=100, rho=0.5)
    print(s)
    print(quadrant(*s))
    print(pearson(*s))
    print(spearman(*s))
    print(mixed(10))
    fig, ax = plt.subplots()
    plotEllipse(*s, ax, 3)
    plt.show()

\end{verbatim}
\hrule width 16cm height 1pt

\newpage

\section{Результаты работы программы}

\subsection{Эллипсы рассеивания}

На рис. \ref{fig:20}-\ref{fig:100} представлены результаты  работы программы. На каждом из рисунков представлены эллипсы рассеивания, построенные по формуле \eqref{ellipse}.

На рис. \ref{fig:20} представлены эллипсы рассеивания для выборок из 20 элементов.

\begin{figure}[H]
	\centering
	\includegraphics[height=0.3\linewidth]{200.png}
	\caption{Эллипсы рассеивания для выборок из 20 элементов.}
	\label{fig:20}
\end{figure}

На рис. \ref{fig:60} представлены эллипсы рассеивания для выборок из 60 элементов.

\begin{figure}[H]
	\centering
	\includegraphics[height=0.3\linewidth]{600.png}
	\caption{Эллипсы рассеивания для выборок из 60 элементов.}
	\label{fig:60}
\end{figure}


На рис. \ref{fig:100} представлены эллипсы рассеивания для выборок из 100 элементов.
\begin{figure}[H]
	\centering
	\includegraphics[height=0.3\linewidth]{1000.png}
	\caption{Эллипсы рассеивания для выборок из 100 элементов.}
	\label{fig:100}
\end{figure}

\subsection{Выборочные коэффициенты корреляции}

Ниже, в таблицах \ref{rho0} - \ref{rho09} представлены выборочные коэффициенты корреляции Пирсона, Спирмена и квадрантный коэффициент корреляции для выборок размером 20, 60 и 100 элементов двумерного нормального распределения $N(x, y, 0, 0, 1, 1, \rho)$ с коэффициентами корреляции $\rho = 0, 0.5, 0.9$. 
\begin{table}[h]
    \begin{center}
        \caption{Выборочные коэффициенты корреляции для двумерного нормального распределения, $\rho = 0$}
        \phantom{0}\\
        \begin{tabular}{||c||c|c|c||c|c|c||c|c|c||}\hline
            \multirow{2}*{$\rho = 0 \,\, \eqref{rho} $}& \multicolumn{3}{c||}{$r\mathstrut$ (\ref{r})} & \multicolumn{3}{c||}{$r_Q$ (\ref{rq})} & \multicolumn{3}{c||}{$r_S$ (\ref{rs})}\\
            \cline{2-10}
                & $E(z)\mathstrut$ & $E(z^2)$ & $D(z)$ & $E(z)$ & $E(z^2)$ & $D(z)$ & $E(z)$ & $E(z^2)$ & $D(z)$\\
            \hline
            $N=20$ & $0.005\mathstrut$ & 0.053 & 0.053 & 0.006 & 0.053 & 0.053 & 0.007 & 0.052 & 0.052\\
            \hline
            $N=60$ & $0.001\mathstrut$ & 0.018 & 0.018 & 0.004 & 0.017 & 0.017 & 0.004 & 0.018 & 0.017\\
            \hline
            $N=100$ & $0.003\mathstrut$ & 0.011 & 0.011 & 0.001 & 0.01 & 0.01 & 0.004 & 0.011 & 0.011\\
            \hline
        \end{tabular}
    \label{rho0}
    \end{center}
\end{table}

\begin{table}[h]
    \begin{center}
        \caption{Выборочные коэффициенты корреляции для двумерного нормального распределения, $\rho = 0.5$}
        \phantom{0}\\
        \begin{tabular}{||c||c|c|c||c|c|c||c|c|c||}\hline
            \multirow{2}*{$\rho = 0.5$}& \multicolumn{3}{c||}{$r\mathstrut$} & \multicolumn{3}{c||}{$r_Q$} & \multicolumn{3}{c||}{$r_S$}\\
            \cline{2-10}
                & $E(z)\mathstrut$ & $E(z^2)$ & $D(z)$ & $E(z)$ & $E(z^2)$ & $D(z)$ & $E(z)$ & $E(z^2)$ & $D(z)$\\
            \hline
            $N=20$ & $0.5\mathstrut$ & 0.3 & 0.03 & 0.324 & 0.155 & 0.05 & 0.46 & 0.24 & 0.035\\
            \hline
            $N=60$ & $0.5\mathstrut$ & 0.26 & 0.01 & 0.32 & 0.12 & 0.016 & 0.5 & 0.24 & 0.01\\
            \hline
            $N=100$ & $0.5\mathstrut$ & 0.26 & 0.005 & 0.33 & 0.12 & 0.009 & 0.5 & 0.24 & 0.006\\
            \hline
        \end{tabular}
    \label{rho05}
    \end{center}
\end{table}

\begin{table}[h]
    \begin{center}
        \caption{Выборочные коэффициенты корреляции для двумерного нормального распределения, $\rho = 0.9$}
        \phantom{0}\\
        \begin{tabular}{||c||c|c|c||c|c|c||c|c|c||}\hline
            \multirow{2}*{$\rho = 0.9$}& \multicolumn{3}{c||}{$r\mathstrut$} & \multicolumn{3}{c||}{$r_Q$} & \multicolumn{3}{c||}{$r_S$}\\
            \cline{2-10}
                & $E(z)\mathstrut$ & $E(z^2)$ & $D(z)$ & $E(z)$ & $E(z^2)$ & $D(z)$ & $E(z)$ & $E(z^2)$ & $D(z)$\\
            \hline
            $N=20$ & $0.9\mathstrut$ & 0.8 & 0.003 & 0.6916 & 0.5064 & 0.03 & 0.9 & 0.75 & 0.005\\
            \hline
            $N=60$ & $0.89\mathstrut$ & 0.8 & 0.0007 & 0.7 & 0.5 & 0.009 & 0.89 & 0.78 & 0.001\\
            \hline
            $N=100$ & $0.9\mathstrut$ & 0.8 & 0.0004 & 0.7042 & 0.5 & 0.005 & 0.9 & 0.8 & 0.0006\\
            \hline
        \end{tabular}
    \label{rho09}
    \end{center}
\end{table}
Ниже в таблице \ref{mixTable} представлены выборочные коэффициенты корреляции Пирсона, Спирмена и квадрантный коэффициент корреляции для выборок смеси двумерных нормальных распределений (\ref{mix}) размером 20, 60 и 100 элементов.
\newpage
\begin{table}[h]
    \begin{center}
        \caption{Выборочные коэффициенты корреляции для смеси двумерных нормальных распределений}
        \phantom{0}\\
        \begin{tabular}{||c||c|c|c||c|c|c||c|c|c||}\hline
            \multirow{2}*{}& \multicolumn{3}{c||}{$r\mathstrut$} & \multicolumn{3}{c||}{$r_Q$} & \multicolumn{3}{c||}{$r_S$}\\
            \cline{2-10}
                & $E(z)\mathstrut$ & $E(z^2)$ & $D(z)$ & $E(z)$ & $E(z^2)$ & $D(z)$ & $E(z)$ & $E(z^2)$ & $D(z)$\\
            \hline
            $N=20$ & $-0.35\mathstrut$ & 0.57 & 0.45 & 0.5332 & 0.32 & 0.036 & 0.47 & 0.3 & 0.08\\
            \hline
            $N=60$ & $-0.65\mathstrut$ & 0.5 & 0.08 & 0.56 & 0.32 & 0.01 & 0.5 & 0.25 & 0.026\\
            \hline
            $N=100$ & $-0.7\mathstrut$ & 0.52 & 0.03 & 0.56 & 0.32 & 0.007 & 0.47 & 0.23 & 0.02\\
            \hline
        \end{tabular}
    \label{mixTable}
    \end{center}
\end{table}

\newpage
 \section*{Заключение}
{\addcontentsline {toc}{section}{Заключение}}
\par В рамках лабораторной работы были сгенерированы выборки разного размера для двумерного нормального распределения сс коэффициентами корреляции $\rho = 0, 0.5, 0.9$  и для смеси двумерных нормальных распределений.  Для полученных выборок были рассчитаны выборочные коэффициенты корреляции Пирсона и Спирмена, квадрантный коэффициент корреляции. Так же для выборок были построены эллипсы рассеивания.


При изучении построенных эллипсов рассеивания было установлено, что чем больше коэффициент корреляции, тем больше сужается эллипс рассеивания, в пределе вырождаясь в прямую, при модуле коэффициента корреляции равном единице. Для выборки размером 100 элементов с нулевой корреляцией эллипс очень близок к окружности. 

Для таких выборок видно, что коэффициент корреляции Пирсона очень близок к коэффициенту корреляции выборки. Коэффициент корреляции Спирмена менее точен, но тоже достаточно близок к коэффициенту корреляции выборки.
Исходя из оценок выборочных коэффициентов корреляции можно сделать вывод, что квадрантный выборочный коэффициент имеет наибольшее отклонение от теоретического значения коэффициента корреляции, но с увеличением мощности выборки все выборочные коэффициенты корреляции стремятся к своим теоретическим значениям.\\


\newpage

\addcontentsline{toc}{section}{Список используемой литературы}
\begin{thebibliography}{}
	\bibitem{lit1}  Вероятностные разделы математики. Учебник для бакалавров технических направлений.// Под ред. Максимова Ю.Д. — Спб.: «Иван Федоров», 2001. — 592 c., илл.
	\bibitem{lit2}  Вентцель Е.С. Теория вероятностей: Учеб. для вузов. — 6-е изд. стер. — М.: Высш. шк., 1999.— 576 c.
\end{thebibliography}
\newpage
\appendix


\section{Приложение}
\label{sec:A}
\par Исходный код программы размещен на сервисе GitHub.
\par Ссылка на репозиторий: \url{https://github.com/foria1405/TVlabs}
\end{document}