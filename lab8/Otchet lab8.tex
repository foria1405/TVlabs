\documentclass[12pt]{article}
\usepackage[utf8]{inputenc}
\usepackage[russian]{babel}
\usepackage [left=30 mm, top=30 mm, right=30 mm, bottom=20mm, nohead, footskip=10 mm] {geometry}
\usepackage{pscyr}
\usepackage[T2A]{fontenc}
\usepackage{amsmath}
\usepackage{multirow}
\usepackage{xcolor}
\usepackage{amssymb}
\usepackage{graphicx}
\graphicspath{{src/}}
\usepackage{listings}   
\usepackage{hyperref}
\usepackage{fancyhdr}
%\usepackage{algorithm}
\usepackage{algpseudocode}
\usepackage{indentfirst}
\usepackage{listings}
\usepackage{float}%"Плавающие" картинки
\hypersetup{
    colorlinks=true,
    linkcolor=blue,
    filecolor=magenta,      
    urlcolor=cyan,
    pdftitle={Sharelatex Example},
    bookmarks=true,
    pdfpagemode=FullScreen,
}
\usepackage{wrapfig}%Обтекание фигур (таблиц, картинок и прочего)

\parindent=24pt


\begin{document}

\begin{center}
\hfill \break
\large{МИНОБРНАУКИ РОССИИ} \\
\hfill \break
\small {ФЕДЕРАЛЬНОЕ ГОСУДАРСТВЕННОЕ БЮДЖЕТНОЕ ОБРАЗОВАТЕЛЬНОЕ УЧРЕЖДЕНИЕ }\\
\small { ВЫСШЕГО ПРОФЕССИОНАЛЬНОГО ОБРАЗОВАНИЯ  } \\
\hfill \break
\normalsize {\textbf{ <<САНКТ-ПЕТЕРБУРГСКИЙ ПОЛИТЕХНИЧЕСКИЙ УНИВЕРСИТЕТ } }\\
{\normalsize {\textbf { ПЕТРА ВЕЛИКОГО>>}}} \\
\hfill \break
\large{Институт Прикладной математики и механики }\\
\hfill \break
\large{ Кафедра: <<Телематика ( при ЦНИИ РТК )>> }\\
\hfill \break
Направление 02.03.01 Математика и компьютерные науки\\
\vskip 1cm
\Large {Отчёт по дисциплине:}
\vskip 0.2cm
\Large{<<Теория вероятностей и математическая статистика >>} \\
\hfill \break
\large{Лабораторная работа № 8} \\
\hfill \break
\large{<<Доверительные интервалы для параметров нормального распределения>>} \\
\hfill \break
\vskip 0.3cm
\vskip 0.5cm
\end{center}


\begin {tabular}{cccc}
\hspace{0.5cm}Обучающийся: &\underline {\hspace{3cm}} &  &Фомина Дарья Дмитриевна \\\\
\hspace{0.5cm}Руководитель: &\underline {\hspace{3cm}} & &Баженов Александр Николаевич\\\\
\end{tabular}
\vskip 1.5 cm
\hspace{9cm}\def \hrf#1{\hbox to#1{\hrulefill}}<<\hrf{2em}>>  \hrf{6em}  20\hrf{1em}~r.
\vskip 1.5cm
\begin {center} Санкт-Петербург 2021 \end{center}

\thispagestyle{empty}

\newpage


\tableofcontents


\newpage
\section{Постановка задачи}

\begin{itemize}
\item Необходимо сгенерировать выборки размером 20 и 100 элементов для нормального распределения N(0,1). 
\item Для полученных выборок с помощью метода максимального правдоподобия найти  оценки параметров $\mu$  и $\sigma$. 
\item На их основе найти интервальные оценки асимптотически и на основе статистик Стьюдента и $\chi^2$.
\end{itemize}
В качестве параметра надежности необходимо взять $\gamma = 0.95$

\newpage	

	\section{Математическое описание}
	
	\subsection{Доверительные интервалы для параметров нормального распределения}

\subsubsection{Доверительный интервал для матожидания $m$ нормального распределения}

Для выборки $(x_1, ..., x_n)$ из нормальной генеральной совокупности найдём среднее $\overline{x}$ и среднее квадратичное отклонение $s$.
\vskip 0.3cm
Тогда величина 

\begin{equation}
	T = \sqrt{n - 1} \cdot \frac{\overline{x} - m}{s},
\end{equation}

называемая {\it статистикой Стьюдента}, распределена по закону Стьюдента с $n-1$ степенями свободы.
\vskip 0.3cm
Произведя несложные преобразования, получим, что:

\begin{equation}
P \left( -x < T < x \right) = 2F_T(x) - 1,
\end{equation}

где $F_T$ -- функция распределения Стьюдента с $n-1$ степенями свободы.
\vskip 0.3cm
Полагая $2F_T(x) - 1 = 1 - \alpha$, где $\alpha$ -- уровень значимости, имеем:

\begin{multline}\label{mStud}
\displaystyle P \left( \overline{x} - \frac{sx}{\sqrt{n-1}} < m < \overline{x} + \frac{sx}{\sqrt{n-1}} \right) = \\
= P \left( \overline{x} - \frac{st_{1 - \frac{\alpha}{2}}(n - 1)}{\sqrt{n-1}} < m < \overline{x} + \frac{st_{1 - \frac{\alpha}{2}}(n - 1)}{\sqrt{n-1}} \right) = 1 - \alpha
\end{multline}
\vskip 0.3cm
И таким образом получаем доверительный интервал для матожидания с вероятностью $1 - \alpha$. \cite{lit1}

\subsection{Доверительный интервал для среднего квадратического отклонения $\sigma$ нормального распределения}

Доказано, что случайная величина $\frac{ns^2}{\sigma^2}$ распределена по закону $\chi^2$ с $n-1$ степенями свободы. 

После ряда преобразований, получаем:

\begin{equation}
	\displaystyle P \left( \frac{s \sqrt{n}} {\sqrt{\chi_{1 - \alpha/2}^2(n-1)}} < \sigma <  \frac{s \sqrt{n}} {\sqrt{\chi_{\alpha/2}^2(n-1)}} \right)
	\label{s_norm}
\end{equation}
\vskip 0.3cm
что и даёт доверительный интервал для $\sigma$ с доверительной вероятностью $\gamma = 1 - \alpha$. \cite{lit1}

\subsection{Доверительные оценки для параметров произвольного распределения. Асимптотический подход}

\subsubsection{Доверительные оценки для матожидания при большом размере выборки}

Если исследуемое распределение имеет конечное матожидание и дисперсию, то имеет место центральная предельная теорема:

\begin{equation}
\frac{\overline{x}-\mathbf{M}x}{\sqrt{\mathbf{Dx}}}=\sqrt{n} \cdot \frac{\overline{x} - m}{\sigma} \overset{F}{\longrightarrow} N(x, 0, 1)
\end{equation}

Отсюда получаем, что

\begin{equation*}
P \left(-x < \sqrt{n} \cdot \frac{\overline{x} - m}{\sigma} < x \right) \approx 2 \Phi(x),
\end{equation*}

где $\Phi(x)$ -- функция Лапласа.
\vskip 0.3cm
Полагая $u_{1 - \alpha / 2}$ за соответствующий квантиль центрированного нормального распределения с единичной дисперсией, получаем:

\begin{equation}\label{m_pr}
P \left(\overline{x} - \frac{su_{1 - \alpha / 2}}{\sqrt{n}} < m < \overline{x} + \frac{su_{1 - \alpha / 2}}{\sqrt{n}} \right) \approx \gamma,
\end{equation}

что и даёт доверительный интервал для матожидания $m$ с доверительной вероятностью $\gamma$. \cite{lit1}


\subsubsection{Доверительный интервал для среднего квадратического отклонения $\sigma$ произвольной генеральной совокупности при большом объёме выборки}
Предполагаем, что исследуемая генеральная совокупность имеет конечные первые четыре момента.\\
\vskip 0.1cm
$u_{1-\alpha/2}$ -- квантиль нормального распределения $N(x, 0, 1)$ порядка $1-\alpha/2$.\\
\vskip 0.1cm
$E = \frac{\mu_4}{\sigma^4} - 3$ -- эксцесс генерального распределения, $e = \frac{m_4}{s^4} - 3$ -- выборочный эксцесс; 
\par $m_4 = \frac{1}{n}\sum_{i=1}^{n}(x_i - \overline{x})^4$ -- четвёртый выборочный центральный момент.

\begin{equation}s(1 + U)^{-1/2} < \sigma < s(1 - U)^{-1/2}, 
\label{s_pr}
\end{equation}
или 

\begin{equation}s(1 - 0.5U) < \sigma < s(1 + 0.5U),
\label{sAsympt}
\end{equation}
где $U = u_{1-\alpha/2}\sqrt{(e+2)/n}$ 
\vskip 0.3cm
Приведенные формулы дают доверительный интервал для $\sigma$ с доверительной вероятностью $\gamma = 1-\alpha$. \cite{lit1}




\newpage
	\section{Особенности реализации}
	
		Для генерации выборок, вычисления квантилей нормального распределения, распределения Стьюдента и распределения хи-квадрат в  данной лабораторной работе была использована библиотека scipy.  
		
		Так же были реализованы функции, выполняющие вычисления интервальных оценок в соответствии с формулами (\ref{s_norm})-(\ref{sAsympt}).
\vskip 0.3cm
\hrule width 16cm height 1pt
\begin{verbatim}
   print('Классические интервальные оценки')
    dx = s*ct*(n - 1)**(-0.5)
    print(f'm in ({xm-dx}; {xm+dx})')
    print(f's in ({s*(n/chi_low)**(0.5)}; {s*(n/chi_high)**(0.5)})')
\end{verbatim}
\vskip 0.3cm
\hrule width 16cm height 1pt
\vskip 0.3cm
\begin{verbatim}
print('Асимптотические интервальные оценки')
    cu = norm.ppf((1 + gamma) / 2)
    dx = s * cu *(n**(-0.5))
    m4 = np.mean((sel - xm)**4)
    e = m4/(s**4) - 3
    U = cu*np.sqrt((e+2)/n)
    print(f'm in ({xm-dx}; {xm+dx})')
    print(f's in ({s*(1+U)**(-0.5)}; {s*(1-U)**(-0.5)})')
\end{verbatim}
\hrule width 16cm height 1pt
\newpage

\section{Результаты работы программы}
В таблице \ref{table:chi} представлены интервальные оценки на основе статистик Стьюдента и хи-квадрат.

\begin{table}[H]
	\begin{center}
		\begin{tabular}{|l|l|l|}
		\hline
		$n$ & Интервал $\mu$  (\ref{mStud})  & Интервал $\sigma$ (\ref{s_norm})\\ \hline
		20  & $(-0.34; 0.63)$ & $(0.79, 1.50)$  \\ \hline
		100 & $(-0.17; 0.21)$ & $(0.83; 1.12)$  \\ \hline
	\end{tabular}
	\end{center}
\caption{Интервальные оценки на основе статистик Стьюдента и хи-квадрат}

\label{table:chi}
\end{table}
	
	 В таблице \ref{table:asumpt} представлены асимптотические интервальные оценки.
	
	
	\begin{table}[H]
		\begin{center}
			\begin{tabular}{|l|l|l|}
			\hline
			$n$ & Интервал $\mu$ (\ref{m_pr})   & Интервал $\sigma$ (\ref{sAsympt})\\ \hline
			20  & $(-0.29; 0.59)$ & $(0.82; 1.41)$  \\ \hline
			100 & $(-0.04; 0.35)$ & $(0.85; 1.09)$  \\ \hline
		\end{tabular}
		\end{center}
	
	\caption{Асимптотические интервальные оценки}
	\label{table:asumpt}
	\end{table}


\newpage
 \section*{Заключение}
{\addcontentsline {toc}{section}{Заключение}}
\par  В ходе данной лабораторной работы для были сгенерированы выборки размером 20 и 100 элементов для нормального распределения. С помощью метода максимального правдоподобия для полученных выборок  были найдены оценки параметров $\mu$ И  $\sigma$. На основе полученных оценок были найдены интервальные оценки на основе статистик Стьдента и хи-квадрат, так же интервальные оценки были найдены асимптотически.
	
	При изучении полученных оценок можно сделать вывод о том, что асимптотический метод оценки доверительного интервала демонстрирует более точные результаты в отличии от оценок на основе статистик Стьдента и хи-квадрат. 
	
	Также заметно, что при увеличении размера выборки, доверительные интервалы становятся меньшими по длине, то есть более точными.

На основании полученных результатов можно сделать вывод о том, что длина доверительных интервалов уменьшается с увеличением выборки, то есть возрастает точность. Доверительные интервалы для параметров нормального распределения более надёжны, так как основаны на точном, а не асимптотическом распределении.

\newpage





\addcontentsline{toc}{section}{Список используемой литературы}
\begin{thebibliography}{}
	\bibitem{lit1}  Вероятностные разделы математики. Учебник для бакалавров технических направлений.// Под ред. Максимова Ю.Д. — Спб.: «Иван Федоров», 2001. — 592 c., илл.
	\bibitem{lit2}  Вентцель Е.С. Теория вероятностей: Учеб. для вузов. — 6-е изд. стер. — М.: Высш. шк., 1999.— 576 c.
\end{thebibliography}
\newpage
\appendix


\section{Приложение}
\label{sec:A}
\par Исходный код программы размещен на сервисе GitHub.
\par Ссылка на репозиторий: \url{https://github.com/foria1405/TVlabs}
\end{document}

\newpage
\section{Математическое описание}